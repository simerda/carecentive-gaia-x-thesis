\chapter{Results}\label{ch:results}

\begin{chapterabstract}
    This chapter features the results of the experiments that were conducted, which are presented in Chapter~\ref{ch:methods}.
    For clarity, the experiment evaluations are discussed separately in their sections.
    The experiments examine the working state and functionality of the features in the Gaia-X Ecosystem.
\end{chapterabstract}

\section[Participant Compliance]{Obtaining Gaia-X compliance for Participant-related Credentials}\label{sec:obtaining-gaia-x-compliance-for-participant-related-credentials}

\textbf{Result}: \textit{Pass}
\\
\textbf{Comments}: \texttt{LegalRegistrationNumber}, \texttt{TermsAndConditions}, and \texttt{Participant} Gaia-X Credentials were successfully created, and their compliance was obtained.

\section[Service Offering Compliance]{Obtaining Gaia-X compliance for ServiceOffering-related Credentials}\label{sec:obtaining-gaia-x-compliance-for-serviceoffering-related-credentials}

\textbf{Result}: \textit{Fail}
\\
\textbf{Comments}: \texttt{ServiceOffering}, \texttt{DataResource}, \texttt{SoftwareResource}, \texttt{InstantiatedVirtualResource}, and \texttt{ServiceAccessPoint} Gaia-X Credentials were created successfully, but the Gaia-X Compliance returned an error.
The exact response from Compliance can be seen in Figure~\ref{fig:service_offering_compliance_response}.
The response indicates a user error, which the response body narrows down to the \texttt{exposedThrough} attribute of the \texttt{DataResource} Credential.

Upon examining the Gaia-X Registry, the database of the trusted credential shapes mandates the value to be a link to a Credential of type \texttt{ServiceOffering}.
However, the \texttt{DataResource} Credential created by the Carecentive backend links a credential of type \texttt{InstantiatedVirtualResource}, which is in line with the Gaia-X Trust Framework specifications~\cite{gaiax_trust_framework}.
The mismatch between documentation requirements and the formally implemented vocabulary is probably the reason for failure.

\begin{figure}[h]
    \centering
    \begin{minted}[bgcolor=LightGray,breaklines]{http}
HTTP/1.1 409 Conflict
Content-Type: application/json

{
	"statusCode": 409,
	"message": {
		"conforms": false,
		"results": [
			"ERROR: https://gaia-x-dev.simerda.dev/gaia-x/acme-corp/241de422-e7af-4ed4-8e30-2f6f07726b19/data-resource.json#cs https://registry.lab.gaia-x.eu/development/api/trusted-shape-registry/v1/shapes/jsonld/trustframework#exposedThrough: "
		]
	},
	"error": "Conflict"
}
    \end{minted}
    \caption{The response from the Gaia-X Compliance service upon trying to obtain compliance for ServiceOffering-related credentials.}\label{fig:service_offering_compliance_response}
\end{figure}

\section[Data Product Compliance]{Obtaining Gaia-X compliance for DataProduct-related Credentials}\label{sec:obtaining-gaia-x-compliance-for-dataproduct-related-credentials}

\textbf{Result}: \textit{Fail}
\\
\textbf{Comments}: \texttt{DataProductDescription}, \texttt{DatasetDescription}, and \texttt{DataUsage} Gaia-X Credentials were created successfully, but the Gaia-X Compliance returned an error.
The exact response from Compliance can be seen in Figure~\ref{fig:data_product_compliance_response}.
The response indicates a \textit{user error}, which the response body narrows down to a shape not being recognized by the Registry.
The error implies that the Gaia-X credentials defined by the Gaia-X ``\textit{Data Exchange specifications}''~\cite{gaiax_data_exchange_document} have not yet been implemented in the running GXDCH services.

\begin{figure}[h]
	\centering
	\begin{minted}[bgcolor=LightGray,breaklines]{http}
HTTP/1.1 409 Conflict
Content-Type: application/json

{
	"message": "VerifiableCrdential contains a shape that is not defined in registry shapes",
	"error": "Conflict",
	"statusCode": 409
}
	\end{minted}
	\caption{The response from the Gaia-X Compliance service upon trying to obtain compliance for DataProduct-related credentials.}\label{fig:data_product_compliance_response}
\end{figure}

\section{Exchange of questionnaire data}\label{sec:exchange-of-questionnaire-data}

\textbf{Result}: \textit{Pass}
\\
\textbf{Comments}: The data exchange contracting process is carried out successfully, resulting in a valid contract.
The contract is then used to successfully issue the access token, upon which the consumer successfully fetches the questionnaire data.

\section{Exchange of Withings data}\label{sec:exchange-of-withings-data}

\textbf{Result}: \textit{Pass}
\\
\textbf{Comments}: The data exchange contracting process is carried out successfully, resulting in a valid contract.
The contract is then used to successfully issue the access token, upon which the consumer successfully fetches the Withings data.

\section{Data exchange with policy negotiation}\label{sec:data-exchange-with-policy-negotiation}

\textbf{Result}: \textit{Unable to Conduct}
\\
\textbf{Reason}: \textit{Missing XFSC component}
\\
\textbf{Comments}: The experiment cannot be conducted because the ``\textit{Data Contracting Services}'' component is missing from the XFSC reference implementation.
The component is still under development but should provide policy negotiation and contracting services once it is finished.

\section{Summary}\label{sec:summary}

As previous sections showcase, the outcome of the experiments is the following.

\begin{longtable}{ |l|c|c|c| }
	\hline
	\textbf{Experiment} & \textbf{Unable to Conduct} & \textbf{Pass} & \textbf{Fail}\\
	\hline
	\texttt{Participant Compliance} &&X&\\
	\hline
	\texttt{ServiceOffering} Compliance &&&X\\
	\hline
	\texttt{DataProduct} Compliance &&&X\\
	\hline
	Questionnaire Data Exchange &&X&\\
	\hline
	Withings Data Exchange &&X&\\
	\hline
	Data Exchange with Policy Negotiation &X&&\\
	\hline
	\textbf{Sum} &&3&2\\
	\hline
	\caption{Overview of the results of the experiments that were conducted.}
	\label{tab:results}
\end{longtable}
