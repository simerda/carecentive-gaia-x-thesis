\chapter{Introduction}\label{ch:introduction}

\section{Motivation}\label{sec:motivation}

The ongoing developments of hardware and software in information technology have caused electronic devices to steadily become more compact, lighter,
and more affordable for end customers.
Coupled with advancements in battery life, this trend has enabled the emergence of smart wearable devices,
with the first-ever market-produced ``smartwatches''
\footnote{The \textit{smartness} of the "Pulsar Calculator Watch" though was vastly different from our current
understanding of smart gadgets and was limited to arithmetic calculations.} appearing as early as 1975\cite{ometov_survey_2021}.
Over time, this has ignited a growing interest in these devices among the general public, with now around
30\%\cite{simon_kemp_rise_2023} of internet users owning a smartwatch or a smart wristband.

The hype around consumer-available wearable devices didn't leave the healthcare industry untouched.
Even though digital technologies have been successfully used in clinical for many years, the recent advances in
hardware and analytic capabilities have caused an explosion of interest in the use of digital technologies to
facilitate data collection in clinical trials\cite{clay_impact_2017}.

The use of wearable technology in medical trials can potentially offer numerous benefits compared to the
traditional data collection approach.
These include --- but are not limited to --- longitudinal data measurement, out of clinic use, reduced cost, treatment response monitoring and improved data quality\cite{munos_mobile_2016}.

The deployment of smart wearable devices in clinical trials along with their continuous recording capabilities results in significant increases in data volume\cite{munos_mobile_2016}.
Which may extend the possibilities of using deep learning, where the requirement for the size of a training dataset is increased, to analyze this data.

The increase in generated data volume presents an opportunity to exchange this data with other researchers or institutions, which can further reduce the cost of research.

This thesis examines the feasibility of using a Gaia-X-compliant solution to perform this exchange.
\textit{Gaia-X}\cite{gaiax} is an initiative that emerged in 2019 in the EU and whose goal is to govern how data, digital services and infrastructure are exchanged in a decentralized federated environment.
They do this by publishing specifications, parties must follow in order to participate.
Alongside the specifications, they also provide a reference implementation of the software services needed to run a Gaia-X federation under the project name Gaia-X Federation Services\cite{gxfs}.
This includes mainly the framework facilitating data and software transactions and other useful tools, for example, a credential manager or data exchange logger.

\section{Purpose}\label{sec:purpose}

While the initiative already emerged in 2019, saw a continued increase in the number of members, lighthouse projects and national hubs, and was made publicly available in 2023; there have also been reports of implementation delays, lack of agility, excessive bureaucracy and concerns about the market readiness of Gaia-X\cite{say_gaia-x_2024},~\cite{noauthor_inside_2021},~\cite{eichberger_why_2021}.
The purpose of this thesis is to assess the production-readiness of Gaia-X and to determine the feasibility of exchanging healthcare data via Gaia-X compliant processes.
This is done through the development of a Gaia-X-compliant module for data exchange in an existing framework Carecentive\cite{noauthor_carecentivenet_nodate}.
The exact research questions this thesis seeks to answer are outlined in Table~\ref{tab:research-questions}.

\begin{table}
    \centering
    {\renewcommand{\arraystretch}{1.7}
        \begin{tabular}{ p{4cm}|p{11cm} }
            Research Question 1 & Is it feasible to share healthcare data in Gaia-X Data Spaces?\\
            \hhline{--}
            Research Question 2 & What are the technical challenges in implementing Gaia-X-compliant software?
        \end{tabular}
    }
    \caption{Research Questions}
    \label{tab:research-questions}
\end{table}

\section{Outline}\label{sec:outline}

% TODO: reference the chapters, verify at the end of work
This section concludes the first chapter, which clarifies the motivation, purpose of this thesis, the research questions and finally the outline.
Next up, we provide the fundamentals of federated data and services architectures in Chapter 2.
The next chapter --- Chapter 3 --- then follows up with a deep dive on literature related to the topic of this thesis.
The Chapter 4 serves to clarify the methods of this work used to answer the research questions stated in this chapter.
The Chapter 5 states the results of this work, which are then discussed in Chapter 6.
Lastly, Chapter 7 wraps up this thesis and provides a conclusion.
