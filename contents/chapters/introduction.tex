\chapter{Introduction}\label{ch:introduction}

\section{Motivation}\label{sec:motivation}

The ongoing developments of hardware and software in information technology have made electronic devices steadily more compact, lighter, and affordable for end customers.
The miniaturization of these devices, coupled with advancements in battery life, this trend has enabled the emergence of smart wearable devices, with the first-ever market-produced ``smartwatches''\footnote{The \textit{smartness} of the ``Pulsar Calculator Watch'', though, was vastly different from our current understanding of smart gadgets and was limited to arithmetic calculations.} appearing as early as 1975~\cite{ometov_survey_2021}.
Over time, this has ignited a growing interest in these devices among the general public, with now around 30\%~\cite{simon_kemp_rise_2023} of internet users owning a smartwatch or a smart wristband.

The hype around consumer-available wearable devices didn't leave the healthcare industry untouched.
Even though digital technologies have been successfully used in clinical settings for many years, recent advances in hardware and analytic capabilities have caused an explosion of interest in using digital technologies to facilitate data collection in clinical trials~\cite{clay_impact_2017}.

The deployment of smart wearable devices in clinical trials, along with their continuous recording capabilities, results in significant increases in data volume~\cite{munos_mobile_2016}.
The rise of gathered data volume may extend the possibilities of using deep learning, where the requirement for the size of a training dataset is increased, to analyze this data.

The increased generated data volume presents an opportunity to exchange this data with other researchers or institutions.
This can offer numerous benefits, like allowing other researchers to be able to reproduce a study (data reuse)~\cite{pasquetto_reuse_2017}.
The data can be used with different models for different purposes, or it can even be combined with other heterogeneous datasets (data integration)~\cite{pasquetto_reuse_2017}.
This may lead to increased efficiency and extract more value from a single dataset compared to the data being just single-use since the dataset can be analyzed from different standpoints and using other methods and come to different conclusions~\cite{pasquetto_reuse_2017}.

This thesis examines the feasibility of using a Gaia-X-compliant solution to perform this exchange.
\textit{Gaia-X}~\cite{gaiax} is an initiative that emerged in 2019 in the EU, and its goal is to govern how data, digital services, and infrastructure are exchanged in a decentralized, federated environment.
This initiative does this by publishing specifications that parties must follow to participate.
Alongside the specifications, they also provide a reference implementation of the software services needed to run a Gaia-X federation under the project name Cross Federation Services Components (formerly known as Gaia-X Federation Services)~\cite{gxfs}.
The reference implementation mainly includes the services needed to operate a federation and facilitate data transactions and also offers other valuable tools, such as a credential manager or data exchange logger.

\section{Purpose}\label{sec:purpose}

While the initiative already emerged in 2019, it saw a continued increase in the number of members, lighthouse projects, and national hubs and was made publicly available in 2023.
There have also been reports of implementation delays, a lack of agility, excessive bureaucracy, and concerns about Gaia-X's market readiness~\cite{say_gaia-x_2024, noauthor_inside_2021, eichberger_why_2021}.
This thesis aims to determine the feasibility of exchanging healthcare data via Gaia-X-compliant processes.
The assessment is done by developing a Gaia-X-compliant module for data exchange in an existing framework, Carecentive~\cite{carecentive}, and then conducting experiments on its implementation.
Table~\ref{tab:research-questions} outlines the research questions this thesis seeks to answer.

\begin{table}
    \centering
    {\renewcommand{\arraystretch}{1.7}
        \begin{tabular}{ p{4cm}|p{11cm} }
            Research Question 1 & Is it feasible to share healthcare data in Gaia-X Data Spaces?\\
            \hhline{--}
            Research Question 2 & What are the technical challenges in implementing Gaia-X-compliant software?
        \end{tabular}
    }
    \caption{Research Questions}
    \label{tab:research-questions}
\end{table}

\section{Outline}\label{sec:outline}

This section concludes the first chapter, which clarifies the motivation and purpose of this thesis, the research questions, and the outline.
Next, the fundamentals of federated data and services architectures are provided in Chapter~\ref{ch:fundamentals}.
The next chapter --- Chapter~\ref{ch:related-work} --- then follows up with a deep dive into literature related to the topic of this thesis.
Chapter~\ref{ch:gaia-x-concepts} elaborates on the concepts used in the Gaia-X ecosystem in greater detail than Chapter~\ref{ch:fundamentals}.
Chapter~\ref{ch:methods} clarifies the methods used in this work to answer the research questions stated in this chapter.
Chapter~\ref{ch:results} states the results of this work, which are then discussed in Chapter~\ref{ch:discussion}.
Lastly, Chapter~\ref{ch:conclusion} wraps up this thesis and provides a conclusion.
