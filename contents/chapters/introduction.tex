\chapter{Introduction}\label{ch:introduction}

\section{Motivation}\label{sec:motivation}

The ongoing developments of hardware and software in information technology have caused electronic devices to steadily become more compact, lighter,
and more affordable for end customers.
Coupled with advancements in battery life, this trend has enabled the emergence of smart wearable devices,
with the first-ever market-produced ``smartwatches''
\footnote{The \textit{smartness} of the "Pulsar Calculator Watch" though was vastly different from our current
understanding of smart gadgets and was limited to arithmetic calculations.} appearing as early as 1975\cite{ometov_survey_2021}.
Over time, this has ignited a growing interest in these devices among the general public, with now around
30\%\cite{simon_kemp_rise_2023} of internet users owning a smartwatch or a smart wristband.

The hype around consumer-available wearable devices didn't leave the healthcare industry untouched.
Even though digital technologies have been successfully used in clinical for many years, the recent advances in
hardware and analytic capabilities have caused an explosion of interest in the use of digital technologies to
facilitate data collection in clinical trials\cite{clay_impact_2017}.

The use of wearable technology in medical trials can potentially offer numerous benefits compared to the
traditional data collection approach.
These include --- but are not limited to --- longitudinal data measurement, out of clinic use, reduced cost, treatment response monitoring and improved data quality. %TODO: add citations

The deployment of smart wearable devices in clinical trials along with their continuous recording capabilities result in significant increases in data volume.
Which may extend the possibilities of using deep learning, where the requirement for the size of a training dataset is increased, to analyze this data.

The increase in generated data volume presents an opportunity to exchange this data with other researchers or institutions, which can further reduce the cost of research.

This thesis examines the feasibility of using a Gaia-X-compatible solution to perform this exchange.
Gaia\-X is an initiative that emerged in the EU and whose goal is to govern how data, digital services and infrastructure are exchanged in a decentralized federated environment.
They do this by publishing specifications, participants must abide by in order to participate.
And they provide a reference implementation for the framework which provides the software services needed to facilitate the transaction of data and other services.

\section{The Goal}\label{sec:the-goal}

The goal of this thesis is to assess the production-readiness of Gaia-X and to determine the feasibility of exchanging healthcare data via Gaia-X compliant processes.
The exact research questions this thesis is asking are outlined in Table~\ref{tab:research-questions}.

\begin{table}
    \centering
    \begin{tabular}{ l|l }
        Research Question 1 & Is it feasible to share healthcare data in Gaia-X Data Spaces?\\
        \hhline{--}
        Research Question 2 & What are the technical challenges in implementing Gaia-X specifications?
    \end{tabular}
    \caption{Research Questions}
    \label{tab:research-questions}
\end{table}
