\chapter{Conclusion}\label{ch:conclusion}

The thesis shows, that while application in various industries and healthcare-related use-cases could be possible in the future versions of Gaia-X as envisioned.
However, currently, the application in non-trivial use cases is hindered by missing implementations, inconsistencies in specifications, poor quality of documentation and misalignment between specifications and GXDCH services.
The success of Gaia-X will depend on the careful balancing of economic, legal, and technological factors.

This work aimed to develop a data exchange module to enable the sharing of medical trial data in data spaces.
First, a review of related literature was performed to gain insight into the concepts of data spaces, their use, and potential issues.
The review identified a wide array of uses, ranging from deployment in European Health Data Spaces (EHDSs) to applications in various industries, such as enabling a platform for Digital Twins (DTs) and Product Lifecycle Management (PLM).
The Gaia-X initiative could be crucial in enabling users to have sovereign control over their data and providers to determine the policies for offered resources.

Conversely, some authors raise concerns about the solution's competitiveness compared to the established infrastructure provided by tech giants.
Others caution that the assertive push for digital sovereignty could threaten the EU's traditional role as a normative power and warn against overstepping the sovereignty line towards protectionism.
The different understanding of the term \textit{digital sovereignty} is also discussed.
Additionally, fragmented policies among EU members were identified as a potential complication to establishing EHDS in practice.

During development, a list of technical and specifications-related issues was identified, which is discussed in the previous chapter.
Afterwards, experiments were conducted on the implementation to verify the feasibility of using Gaia-X to exchange medical trial data in the Gaia-X ecosystem.

The failure of experiments no.~2 and 3, which are concerned with obtaining Gaia-X compliance for data exchange-related Gaia-X Credentials, points to missing implementation and imperfect alignment of implemented services with defined specifications.
The inability to conduct experiment no.~6 demonstrates issues with missing implementations of services crucial for data exchange.
The experiments' results show that crucial concepts and workflows presented in the specifications are missing from the available implementations.

The thesis shows that, as envisioned, future versions of Gaia-X could be used in various industries and healthcare-related use cases.
However, the application in non-trivial use cases is currently hindered by missing implementations, inconsistencies in specifications, poor quality of documentation, and misalignment between specifications and GXDCH services.
The success of Gaia-X will depend on the careful balancing of economic, legal, and technological factors.
