\chapter{Related Work}\label{ch:related-work}

\begin{chapterabstract}
    In this chapter, we review literature related to the topics of federated infrastructure, data spaces and Gaia-X specifically.
\end{chapterabstract} % TODO: expand chapter abstract

\section{Designing Data Spaces}\label{sec:designing-data-spaces}

The book ``Designing Data Spaces -- The Ecosystem Approach to Competitive Advantage'' by Boris Otto et al., explores frameworks and technologies behind data spaces, focusing on initiatives like Gaia-X and the International Data Spaces (IDS)\cite{designing_dataspaces}.
It emphasizes the importance of security and data sovereignty; \textit{``natural person’s or corporate entity’s capability of being entirely self-determined with regard to its data''}~\cite{designing_dataspaces} in the context of data exchange for innovation across sectors.
The book discusses the development, principles, technical, and governance aspects of data spaces, highlighting the need for harmonized approaches and public-private partnerships in Europe.

The Chapter 4 -- Role of Gaia-X in the European Data Space Ecosystem, provides a comprehensive introduction to Gaia-X, explaining its objectives, principles, and the economic and business impact of data.
Gaia-X's role in creating a federated, interoperable and secure data infrastructure for industries such as finance, energy, automotive, and health is examined, showing its broad applicability.
The book is a helpful resource for understanding Gaia-X, offering insights into its impact on secure data exchange and innovation.

A noteworthy section of the chapter is about the measuring of success of Gaia-X.
It mentions the goal of doubling the cloud usage in Europe within 4--5 years as well as the announced contribution of 1.2 B\char"20AC~ by the European Commission for the creation of data spaces.

\section{Security Engineering of Patient-Centered Health Care Information Systems in Peer-to-Peer Environments}\label{sec:security-engineering-of-patient-centered-health-care-information-systems-in-peer-to-peer-environments}

The paper ``Security and Privacy Issues in Peer-to-Peer Patient-Centered Health Care Information Systems'' by Abdullahi Yari et al., published in the \textit{Journal of Medical Internet Research} in November 2021, examines the use of peer-to-peer (P2P) networks in personal health systems (PHSs) to decentralize health data management\cite{security_engineering_p2p_environments}.
This approach aims to empower patients and improve health information exchange efficiency.
The study highlights the advantages of P2P networks, such as eliminating central points of failure, enabling direct resource sharing, and allowing self-organization.

Real-world examples, like the \textit{e-toile} P2P framework in Geneva and the \textit{PEPP-PT} COVID-19 contact tracing system, demonstrate the practical benefits of P2P networks in healthcare.
However, the paper also addresses significant security and privacy challenges, including vulnerabilities to snooping nodes, lack of content verification, network heterogeneity, and risks of disclosing sensitive information.
To mitigate these issues, the authors propose solutions like robust authentication protocols, trust and reputation systems, end-to-end encryption, and mobile agent-based intrusion detection systems.% TODO: consider the relevance of these solutions for Gaia-X

The study underscores the importance of ongoing research to ensure P2P PHSs can achieve their potential while maintaining high security and privacy standards.

\section{European Health Data Space}\label{sec:european-health-data-spac}

The paper ``European Health Data Space—An Opportunity Now to Grasp the Future of Data-Driven Healthcare'' by Denis Horgan et al., published in the \textit{Healthcare} journal explores how the European Health Data Space (EHDS) can revolutionize European healthcare by optimizing the use of health data\cite{european_health_data_space}.
The EHDS aims to enhance healthcare delivery, patient outcomes, and research by improving data accessibility, privacy, and interoperability.
Key points include the EHDS's proposal for better access to health datasets for research and public health authorities.
The authors stress the need for strong data protection measures, compliance with European standards and governance to build public trust.

Key challenges highlighted include the need for supportive legislation, stakeholder engagement, and policies that balance data sharing with privacy concerns.
The EHDS represents a crucial step towards a more integrated and efficient healthcare system in Europe, offering valuable insights for advancing data-driven healthcare innovations.

\section{Federated electronic health records for the European Health Data Space}\label{sec:federated-electronic-health-records-for-the-european-health-data-space}

\section{Digital Sovereignty in the EU: Challenges and Future Perspectives}\label{sec:digital-sovereignty-in-the-eu:-challenges-and-future-perspectives}

\section{Project GAIA-X: A Federated Data Infrastructure as the Cradle of a Vibrant European Ecosystem}\label{sec:project-gaia-x:-a-federated-data-infrastructure-as-the-cradle-of-a-vibrant-european-ecosystem}

\section{GAIA-X and IDS}\label{sec:gaia-x-and-ids}

\section{Federated learning of predictive models from federated Electronic Health Records}\label{sec:federated-learning-of-predictive-models-from-federated-electronic-health-records}

\section{Sovereignly Donating Medical Data as a Patient: A Technical Approach}\label{sec:sovereignly-donating-medical-data-as-a-patient:-a-technical-approach}
