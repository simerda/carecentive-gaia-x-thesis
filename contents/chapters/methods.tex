\chapter{Methods}\label{ch:methods}

\begin{chapterabstract}
    In the following chapter outlines the methodology used to assess the Gaia-X initiative, the reference framework implementation by GXFS and answer the research questions stated in the introductory Chapter~\ref{ch:introduction}.%TODO: replace GXFS by XFSC
    The main method of work is implementation of a Gaia-X-compliant data exchange module, serving as a proof of concept.
    Based on this, the Gaia-X-related technical challenges are noted.
    Additionally, a series of experiments is presented as a way to evaluate the proper functionality of the XFSC services, followed with a section on its evaluation.
    Lastly, we present a selection of risks and metrics identified in the Chapter~\ref{ch:related-work} --- Related Work.
\end{chapterabstract}

\section{About Carecentive}\label{sec:about-carecentive}

Carecentive~\cite{carecentive} is a backend software framework used for supporting health studies and trials, which was developed at FAU MaD Lab.
The framework offers features that are typically required in medical trials, notably the following:
\begin{itemize}
    \item User management (registration, login, password management)
    \item Questionnaire storage \& submission
    \item Activity scheduling (e.g., periodical questionnaire submission)
    \item File upload (e.g., images of medical records)
    \item Wearable devices integration (Withings API, Fitbit API)
    \item User activity analytics
    \item Permission management (Roles such as patients, relatives, physicians, nurses, study managers)
\end{itemize}

\subsection{Tech Stack}\label{subsec:tech-stack}

From the developmental standpoint, the framework is divided into two packages, \textit{carecentive-framework}\footnote{\url{https://github.com/carecentive/carecentive-framework}} which mainly sets up the application, registers and exposes the API on the outside.
The API requests are then handed over to the main package \textit{carecentive-code}\footnote{\url{https://github.com/carecentive/carecentive-core}}, which does the heavy lifting of handling user requests based on its business logic.

The Carecentive is written in the \textit{JavaScript} programming language and is running in the \textit{Node.js} engine.
It is built on top of the \textit{Express} web framework, enabling the RESTful API interface, stores data in the relational MySQL DBMS (Database Management Systems).
The database tables are versioned via database migrations enabled by the\textit{Knex.js} library, and relational data is mapped onto objects (models) using the \textit{Objection.js} library.

%TODO: add references to the libraries
In order to simplify setting up the project on different machines and improve stability, the Docker platform was used, which ensures installation of all dependencies and runs the app in an isolated and reproducible environment, reducing the differences in various operating systems.
%TODO: should I explain this more?

\subsection{Use Cases}\label{subsec:use-cases}

Currently, a smartphone app is used as a frontend for the Carecentive backend counterpart in a running study investigating the health of patients receiving care at the palliative care unit in the \textit{Uniklinikum Erlangen} (University Hospital Erlangen).
Thanks to the extensibility of the Carecentive framework, it was also utilized in the works of other students doing their thesis under the FAU MaD Lab (e.g., Smartphone-based Urinalysis).

%TODO: add screenshots of the app

\section{Implementation}\label{sec:implementation}

In order to assess the functionality of the Gaia-X ecosystem, a partial goal of this thesis is to implement a Gaia-X-compliant module for the exchange of the data stored in the Carecentive app.
During the implementation of the exchange module, serving as a proof of concept, any technical or other issues with the Gaia-X specifications and the XFSC software is noted.
After the implementation is finished, a set of predefined data exchange scenarios is run to verify the correct functionality of data exchange module implementation and mainly whether the Gaia-X specifications and XFSC software are ready for a real-world usage.
This is done with in the specific use-case of the palliative care trial.

The implementation task was to design and implement a workflow, which would allow Carecentive users with the \texttt{admin} role to register the questionnaire and Withings\footnote{Withings is a French company developing smart wearable devices for measuring vital signs, activity and sleep monitoring. Carecentive integrates the Withings API to download the data into its database.} data via Gaia-X-compliant processes to allow data consumption to users outside of Carecentive.
To be specific, the \texttt{``/api/withings''} and \texttt{``/api/questionnaires''} API resources, which are protected by a temporary valid token issued upon user login shall be able to accept a new type of token a non-Gaia-X user obtains upon successfully going through the Gaia-X contracting process.

To achieve the goal of enabling Carecentive users to share data in a Gaia-X-compliant manner, a collection of new resources was developed and exposed on the Carecentive API.
A list of all provided endpoints with their description, authorization and HTTP method is available in Table~\ref{tab:endpoints}; \texttt{``/api''} prefix is common for all endpoints and is omitted for brevity.

\begin{longtable}{ |p{4cm}|p{2cm}|p{2cm}|p{7cm}| }
    \hhline{----}
    \textbf{Endpoint} & \textbf{HTTP Method} & \textbf{Authorization} & \textbf{Description}\\
    \hhline{----}
    \texttt{/admin/gaia-x/participants} & \textbf{\texttt{GET}} & Authenticated Admin & Fetches all registered Gaia-X Participants\\
    \hhline{----}
    \texttt{/admin/gaia-x/participants/:participantId} & \textbf{\texttt{GET}} & Authenticated Admin & Fetches a single Gaia-X Participant\\
    \hhline{----}
    \texttt{/admin/gaia-x/participants} & \textbf{\texttt{POST}} & Authenticated Admin & Registers a Gaia-X Participant and creates related credentials\\
    \hhline{----}
    \texttt{/gaia-x/data-products} & \textbf{\texttt{GET}} & Unrestricted & Fetches all registered data products\\
    \hhline{----}
    \texttt{/gaia-x/data-products/:dataProductId} & \textbf{\texttt{GET}} & Unrestricted & Fetches a single data product\\
    \hhline{----}
    \texttt{/gaia-x/data-products} & \textbf{\texttt{POST}} & Authenticated Admin & Registers a data product and creates related Gaia-X credentials\\
    \hhline{----}
    \texttt{/gaia-x/data-products/:dataProductId/contracts} & \textbf{\texttt{POST}} & Unrestricted & Creates a contract proposal for a data product\\
    \hhline{----}
    \texttt{/gaia-x/data-products/:dataProductId/contracts/:contractId} & \textbf{\texttt{PUT}} & Unrestricted & Publishes a signed contract for the given data product\\
    \hhline{----}
    \texttt{/gaia-x/data-products/:dataProductId/contracts/:contractId} & \textbf{\texttt{GET}} & Contract Signature & Fetches the signed contract from both parties\\
    \hhline{----}
    \texttt{/admin/gaia-x/data-product-contracts} & \textbf{\texttt{GET}} & Authenticated Admin & Fetches all data product contracts (proposals)\\
    \hhline{----}
    \texttt{/admin/gaia-x/data-product-contracts/:dataProductContractId} & \textbf{\texttt{GET}} & Authenticated Admin & Fetches a data product contract (proposal)\\
    \hhline{----}
    \texttt{/admin/gaia-x/data-product-contracts/:dataProductContractId} & \textbf{\texttt{PUT}} & Authenticated Admin & Signs a data product contract proposal\\
    \hhline{----}
    \texttt{/admin/gaia-x/data-product-contracts/:dataProductContractId} & \textbf{\texttt{DELETE}} & Authenticated Admin & Rejects a data product contract proposal\\
    \hhline{----}
    \texttt{/gaia-x/authentication} & \textbf{\texttt{POST}} & Unrestricted & Issues a temporary access token based on a valid contract\\
    \hhline{----}
    \caption{An overview of endpoints implemented into Carecentive along with the HTTP method used, authorization method and the description.}
    \label{tab:endpoints}
\end{longtable}

Now, let's take a look at the workflows of a Carecentive admin registering a participant and a data resource, contracting between the user and admin and finally the consumption of data by the user.

\subsection{Data Product registration}\label{subsec:data-product-registration}
The data resource is registered using the \texttt{``/api/gaia-x/data-products''} resource



\section{Experiments}\label{sec:experiments}



\section{Evaluation}\label{sec:evaluation}

In order to evaluate the actual state of the Gaia-X ecosystem, the following tests will be performed with using our implementation of the Carecentive exchange module.

\begin{enumerate}
    \item Obtaining Gaia-X compliance
    \item Creating Gaia-X credentials for data products
    \item Performing data exchange with questionnaire data
    \item Performing data exchange with withings data
\end{enumerate}

The results of these tests are presented in the following chapter.
