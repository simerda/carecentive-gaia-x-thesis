\chapter{Methods}\label{ch:methods}

\begin{chapterabstract}
    In the following chapter, we outline the methods of work used to assess the Gaia-X initiative, the reference framework implementation by GXFS and answer the research questions stated in the introductory Chapter~\ref{ch:introduction}.
    The main method of work is implementation of a Gaia-X-compliant data exchange module, serving as a proof of concept.
    Based on this, we note the Gaia-X related technical challenges.
    Additionally, we take a closer look on the architecture of Gaia-X and at the project in which we implement the data exchange module.
    Lastly, we present a selection of risks and metrics identified in the previous Chapter~\ref{ch:related-work} --- Related Work.
\end{chapterabstract}

\section{About Carecentive}\label{sec:about-carecentive}

\textcolor{red}{This will be a short section about Carecentive. What it is, what data it contains, what technology it's based on.}

\section{Gaia-X Concepts}\label{sec:gaia-x-concepts}

\textcolor{red}{This section will be about selected Gaia-X topics/architecture in a greater details that in fundamentals. Stuff used in the implementation like federated catalogue, self-descriptions, authentication will be emphasised.}

\section{Implementation}\label{sec:implementation}

\textcolor{red}{This will be probably a shorter section about my concrete implementation of the data exchange module in the context of Carecentive.}

\section{Metrics and Risks}\label{sec:metrics-and-risks}

\textcolor{red}{I'm not yet sure about the name of the section but here I'd like to highlight some risks brought in the Related Work section. Next I'd like to gather some metrics based on historical document about Gaia-X milestones etc. Something that will show for example the market acceptance. Then in the Results chapter, it will be compared to reality.}
